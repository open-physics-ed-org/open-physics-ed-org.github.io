\section{\texorpdfstring{\texttt{scan.py} --- OERForge Static Site Asset
and Content
Scanner}{scan.py --- OERForge Static Site Asset and Content Scanner}}\label{scan.py-oerforge-static-site-asset-and-content-scanner}

\subsection{Overview}\label{overview}

\texttt{scan.py} provides the core logic for scanning site content,
extracting assets, and populating the SQLite database for the OERForge
project. It supports hierarchical TOC parsing, asset extraction from
Markdown, Jupyter Notebooks, and DOCX files, and maintains relationships
between pages, sections, and files. The module is designed for clarity,
extensibility, and ease of debugging, with standardized logging and
comprehensive docstrings.

\begin{center}\rule{0.5\linewidth}{0.5pt}\end{center}

\subsection{Module Docstring}\label{module-docstring}

\begin{Shaded}
\begin{Highlighting}[]
\CommentTok{"""}
\CommentTok{scan.py}
\CommentTok{{-}{-}{-}{-}{-}{-}{-}{-}}
\CommentTok{Static Site Asset and Content Scanner}

\CommentTok{This module provides functions for scanning site content, extracting assets, and populating the SQLite database}
\CommentTok{with page, section, and file records. It supports Markdown, Jupyter Notebooks, and DOCX files, and maintains}
\CommentTok{hierarchical relationships from the Table of Contents (TOC) YAML. All functions are documented and organized for}
\CommentTok{clarity and maintainability. Logging is standardized and all major operations are traced for debugging.}

\CommentTok{Key Features:}
\CommentTok{{-} Batch reading and asset extraction for supported file types}
\CommentTok{{-} Hierarchical TOC walking and database population}
\CommentTok{{-} Asset linking and MIME type detection}
\CommentTok{{-} Section and descendant queries using recursive CTEs}

\CommentTok{Usage:}
\CommentTok{    Import and call scan\_toc\_and\_populate\_db(config\_path) to scan the TOC and populate the database.}
\CommentTok{    Use get\_descendants\_for\_parent() to query section hierarchies.}
\CommentTok{"""}
\end{Highlighting}
\end{Shaded}

\begin{center}\rule{0.5\linewidth}{0.5pt}\end{center}

\subsection{Functions}\label{functions}

\subsubsection{\texorpdfstring{\texttt{log\_event(message,\ level="INFO")}}{log\_event(message, level="INFO")}}\label{log_eventmessage-levelinfo}

\begin{Shaded}
\begin{Highlighting}[]
\KeywordTok{def}\NormalTok{ log\_event(message, level}\OperatorTok{=}\StringTok{"INFO"}\NormalTok{):}
    \CommentTok{"""}
\CommentTok{    Logs an event to both stdout and scan.log in the project root.}
\CommentTok{    Uses Python logging and project{-}standard setup.}
\CommentTok{    """}
\end{Highlighting}
\end{Shaded}

\textbf{Purpose:}\\
Standardizes logging for all scan operations, ensuring traceability in
both the console and log files.

\begin{center}\rule{0.5\linewidth}{0.5pt}\end{center}

\subsubsection{\texorpdfstring{\texttt{batch\_read\_files(file\_paths)}}{batch\_read\_files(file\_paths)}}\label{batch_read_filesfile_paths}

\begin{Shaded}
\begin{Highlighting}[]
\KeywordTok{def}\NormalTok{ batch\_read\_files(file\_paths):}
    \CommentTok{"""}
\CommentTok{    Reads multiple files and returns their contents as a dict: \{path: content\}}
\CommentTok{    Supports markdown (.md), notebook (.ipynb), docx (.docx), and other file types.}
\CommentTok{    """}
\end{Highlighting}
\end{Shaded}

\textbf{Purpose:}\\
Efficiently loads content from a list of files, supporting multiple
formats for downstream asset extraction.

\begin{center}\rule{0.5\linewidth}{0.5pt}\end{center}

\subsubsection{\texorpdfstring{\texttt{read\_markdown\_file(path)}}{read\_markdown\_file(path)}}\label{read_markdown_filepath}

\begin{Shaded}
\begin{Highlighting}[]
\KeywordTok{def}\NormalTok{ read\_markdown\_file(path):}
    \CommentTok{"""}
\CommentTok{    Reads a markdown (.md) file and returns its content as a string.}
\CommentTok{    """}
\end{Highlighting}
\end{Shaded}

\textbf{Purpose:}\\
Loads Markdown content for asset extraction and database population.

\begin{center}\rule{0.5\linewidth}{0.5pt}\end{center}

\subsubsection{\texorpdfstring{\texttt{read\_notebook\_file(path)}}{read\_notebook\_file(path)}}\label{read_notebook_filepath}

\begin{Shaded}
\begin{Highlighting}[]
\KeywordTok{def}\NormalTok{ read\_notebook\_file(path):}
    \CommentTok{"""}
\CommentTok{    Reads a Jupyter notebook (.ipynb) file and returns its content as a dict.}
\CommentTok{    """}
\end{Highlighting}
\end{Shaded}

\textbf{Purpose:}\\
Loads Jupyter notebook content for asset extraction and database
population.

\begin{center}\rule{0.5\linewidth}{0.5pt}\end{center}

\subsubsection{\texorpdfstring{\texttt{read\_docx\_file(path)}}{read\_docx\_file(path)}}\label{read_docx_filepath}

\begin{Shaded}
\begin{Highlighting}[]
\KeywordTok{def}\NormalTok{ read\_docx\_file(path):}
    \CommentTok{"""}
\CommentTok{    Reads a docx file and returns its text content as a string.}
\CommentTok{    Requires python{-}docx to be installed.}
\CommentTok{    """}
\end{Highlighting}
\end{Shaded}

\textbf{Purpose:}\\
Loads DOCX content for asset extraction and database population.

\begin{center}\rule{0.5\linewidth}{0.5pt}\end{center}

\subsubsection{\texorpdfstring{\texttt{batch\_extract\_assets(contents\_dict,\ content\_type,\ **kwargs)}}{batch\_extract\_assets(contents\_dict, content\_type, **kwargs)}}\label{batch_extract_assetscontents_dict-content_type-kwargs}

\begin{Shaded}
\begin{Highlighting}[]
\KeywordTok{def}\NormalTok{ batch\_extract\_assets(contents\_dict, content\_type, }\OperatorTok{**}\NormalTok{kwargs):}
    \CommentTok{"""}
\CommentTok{    Extracts assets from multiple file contents in one pass.}
\CommentTok{    contents\_dict: \{path: content\}}
\CommentTok{    content\_type: \textquotesingle{}markdown\textquotesingle{}, \textquotesingle{}notebook\textquotesingle{}, \textquotesingle{}docx\textquotesingle{}, etc.}
\CommentTok{    Returns a dict: \{path: [asset\_records]\}}
\CommentTok{    """}
\end{Highlighting}
\end{Shaded}

\textbf{Purpose:}\\
Extracts and records all linked assets (images, files, etc.) from a
batch of content files, updating the database and linking files to
pages.

\begin{center}\rule{0.5\linewidth}{0.5pt}\end{center}

\subsubsection{\texorpdfstring{\texttt{extract\_linked\_files\_from\_markdown\_content(md\_text,\ page\_id=None)}}{extract\_linked\_files\_from\_markdown\_content(md\_text, page\_id=None)}}\label{extract_linked_files_from_markdown_contentmd_text-page_idnone}

\begin{Shaded}
\begin{Highlighting}[]
\KeywordTok{def}\NormalTok{ extract\_linked\_files\_from\_markdown\_content(md\_text, page\_id}\OperatorTok{=}\VariableTok{None}\NormalTok{):}
    \CommentTok{"""}
\CommentTok{    Extract asset links from markdown text.}
\CommentTok{    Args:}
\CommentTok{        md\_text (str): Markdown content.}
\CommentTok{        page\_id (optional): Page identifier for DB linking.}
\CommentTok{    Returns:}
\CommentTok{        list: File record dicts for each asset found.}
\CommentTok{    """}
\end{Highlighting}
\end{Shaded}

\textbf{Purpose:}\\
Finds and returns all asset links in Markdown content for database
tracking.

\begin{center}\rule{0.5\linewidth}{0.5pt}\end{center}

\subsubsection{\texorpdfstring{\texttt{extract\_linked\_files\_from\_notebook\_cell\_content(cell,\ nb\_path=None)}}{extract\_linked\_files\_from\_notebook\_cell\_content(cell, nb\_path=None)}}\label{extract_linked_files_from_notebook_cell_contentcell-nb_pathnone}

\begin{Shaded}
\begin{Highlighting}[]
\KeywordTok{def}\NormalTok{ extract\_linked\_files\_from\_notebook\_cell\_content(cell, nb\_path}\OperatorTok{=}\VariableTok{None}\NormalTok{):}
    \CommentTok{"""}
\CommentTok{    Extract asset links from a notebook cell.}
\CommentTok{    Args:}
\CommentTok{        cell (dict): Notebook cell.}
\CommentTok{        nb\_path (str, optional): Notebook file path.}
\CommentTok{    Returns:}
\CommentTok{        list: File record dicts for each asset found.}
\CommentTok{    """}
\end{Highlighting}
\end{Shaded}

\textbf{Purpose:}\\
Finds and returns all asset links in notebook cells, including embedded
images and code-generated outputs.

\begin{center}\rule{0.5\linewidth}{0.5pt}\end{center}

\subsubsection{\texorpdfstring{\texttt{extract\_linked\_files\_from\_docx\_content(docx\_path,\ page\_id=None)}}{extract\_linked\_files\_from\_docx\_content(docx\_path, page\_id=None)}}\label{extract_linked_files_from_docx_contentdocx_path-page_idnone}

\begin{Shaded}
\begin{Highlighting}[]
\KeywordTok{def}\NormalTok{ extract\_linked\_files\_from\_docx\_content(docx\_path, page\_id}\OperatorTok{=}\VariableTok{None}\NormalTok{):}
    \CommentTok{"""}
\CommentTok{    Extract asset links from a DOCX file.}
\CommentTok{    Args:}
\CommentTok{        docx\_path (str): Path to DOCX file.}
\CommentTok{        page\_id (optional): Page identifier for DB linking.}
\CommentTok{    Returns:}
\CommentTok{        list: File record dicts for each asset found.}
\CommentTok{    """}
\end{Highlighting}
\end{Shaded}

\textbf{Purpose:}\\
Finds and returns all asset links and embedded images in DOCX files for
database tracking.

\begin{center}\rule{0.5\linewidth}{0.5pt}\end{center}

\subsubsection{\texorpdfstring{\texttt{populate\_site\_info\_from\_config(config\_filename=\textquotesingle{}\_config.yml\textquotesingle{})}}{populate\_site\_info\_from\_config(config\_filename=\textquotesingle\_config.yml\textquotesingle)}}\label{populate_site_info_from_configconfig_filename_config.yml}

\begin{Shaded}
\begin{Highlighting}[]
\KeywordTok{def}\NormalTok{ populate\_site\_info\_from\_config(config\_filename}\OperatorTok{=}\StringTok{\textquotesingle{}\_config.yml\textquotesingle{}}\NormalTok{):}
    \CommentTok{"""}
\CommentTok{    Populate the site\_info table from the given config file (default: \_config.yml).}
\CommentTok{    Args:}
\CommentTok{        config\_filename (str): Name of the config file (e.g., \textquotesingle{}\_config.yml\textquotesingle{}).}
\CommentTok{    """}
\end{Highlighting}
\end{Shaded}

\textbf{Purpose:}\\
Reads site metadata from the config YAML and updates the site\_info
table in the database.

\begin{center}\rule{0.5\linewidth}{0.5pt}\end{center}

\subsubsection{\texorpdfstring{\texttt{get\_possible\_conversions(extension)}}{get\_possible\_conversions(extension)}}\label{get_possible_conversionsextension}

\begin{Shaded}
\begin{Highlighting}[]
\KeywordTok{def}\NormalTok{ get\_possible\_conversions(extension):}
    \CommentTok{"""}
\CommentTok{    Get possible conversion flags for a given file extension.}
\CommentTok{    Args:}
\CommentTok{        extension (str): File extension (e.g., \textquotesingle{}.md\textquotesingle{}, \textquotesingle{}.ipynb\textquotesingle{}).}
\CommentTok{    Returns:}
\CommentTok{        dict: Conversion capability flags.}
\CommentTok{    """}
\end{Highlighting}
\end{Shaded}

\textbf{Purpose:}\\
Determines which conversion operations are supported for a given file
type.

\begin{center}\rule{0.5\linewidth}{0.5pt}\end{center}

\subsubsection{\texorpdfstring{\texttt{scan\_toc\_and\_populate\_db(config\_path)}}{scan\_toc\_and\_populate\_db(config\_path)}}\label{scan_toc_and_populate_dbconfig_path}

\begin{Shaded}
\begin{Highlighting}[]
\KeywordTok{def}\NormalTok{ scan\_toc\_and\_populate\_db(config\_path):}
    \CommentTok{"""}
\CommentTok{    Walk the TOC from the config YAML, read each file, extract assets/images, and populate the DB with content and asset records.}
\CommentTok{    Maintains TOC hierarchy and section relationships.}
\CommentTok{    Args:}
\CommentTok{        config\_path (str): Path to the config YAML file.}
\CommentTok{    """}
\end{Highlighting}
\end{Shaded}

\textbf{Purpose:}\\
The main entry point for scanning the TOC, reading files, extracting
assets, and populating the database with hierarchical content and file
records.

\begin{center}\rule{0.5\linewidth}{0.5pt}\end{center}

\subsubsection{\texorpdfstring{\texttt{get\_descendants\_for\_parent(parent\_output\_path,\ db\_path)}}{get\_descendants\_for\_parent(parent\_output\_path, db\_path)}}\label{get_descendants_for_parentparent_output_path-db_path}

\begin{Shaded}
\begin{Highlighting}[]
\KeywordTok{def}\NormalTok{ get\_descendants\_for\_parent(parent\_output\_path, db\_path):}
    \CommentTok{"""}
\CommentTok{    Query all children, grandchildren, and deeper descendants for a given parent\_output\_path using a recursive CTE.}
\CommentTok{    Args:}
\CommentTok{        parent\_output\_path (str): Output path of the parent section.}
\CommentTok{        db\_path (str): Path to the SQLite database.}
\CommentTok{    Returns:}
\CommentTok{        list: Dicts for each descendant (id, title, output\_path, parent\_output\_path, slug, level).}
\CommentTok{    """}
\end{Highlighting}
\end{Shaded}

\textbf{Purpose:}\\
Efficiently queries the database for all descendants of a section,
supporting hierarchical navigation and index generation.

\begin{center}\rule{0.5\linewidth}{0.5pt}\end{center}

\subsection{Typical Workflow}\label{typical-workflow}

\begin{enumerate}
\def\labelenumi{\arabic{enumi}.}
\tightlist
\item
  \textbf{Scan and Populate Database}

  \begin{itemize}
  \tightlist
  \item
    Call \texttt{scan\_toc\_and\_populate\_db(config\_path)} to walk the
    TOC, read files, extract assets, and populate the database.
  \end{itemize}
\item
  \textbf{Asset Extraction}

  \begin{itemize}
  \tightlist
  \item
    Use \texttt{batch\_extract\_assets} and related helpers to extract
    and link assets from Markdown, Notebooks, and DOCX files.
  \end{itemize}
\item
  \textbf{Site Metadata}

  \begin{itemize}
  \tightlist
  \item
    Use \texttt{populate\_site\_info\_from\_config} to update site-wide
    metadata in the database.
  \end{itemize}
\item
  \textbf{Section Hierarchy Queries}

  \begin{itemize}
  \tightlist
  \item
    Use \texttt{get\_descendants\_for\_parent} to retrieve all children
    and grandchildren for a given section, supporting navigation and
    index pages.
  \end{itemize}
\end{enumerate}

\begin{center}\rule{0.5\linewidth}{0.5pt}\end{center}

\subsection{Logging and Debugging}\label{logging-and-debugging}

All major operations are logged using \texttt{log\_event}, with output
to both stdout and \texttt{scan.log}. This ensures traceability for
debugging and auditing. Errors, warnings, and key events are clearly
marked.

\begin{center}\rule{0.5\linewidth}{0.5pt}\end{center}

\subsection{Extensibility}\label{extensibility}

\begin{itemize}
\tightlist
\item
  New asset types and file formats can be added by extending the asset
  extraction helpers.
\item
  The TOC walking logic supports arbitrary nesting and slug overrides.
\item
  All functions are documented and organized for easy onboarding and
  contribution.
\end{itemize}

\begin{center}\rule{0.5\linewidth}{0.5pt}\end{center}

\subsection{Intended Audience}\label{intended-audience}

\begin{itemize}
\tightlist
\item
  Contributors to OERForge
\item
  Developers building or extending static site generators
\item
  Anyone needing robust asset and content scanning for static sites
\end{itemize}

\begin{center}\rule{0.5\linewidth}{0.5pt}\end{center}
