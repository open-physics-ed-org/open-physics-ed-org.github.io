\section{oerforge.scan}\label{oerforge.scan}

Static Site Asset and Content Scanner for OERForge

\begin{center}\rule{0.5\linewidth}{0.5pt}\end{center}

\subsection{Overview}\label{overview}

\texttt{oerforge.scan} provides functions for scanning site content,
extracting assets, and populating the SQLite database with page,
section, and file records. It supports Markdown, Jupyter Notebooks, and
DOCX files, and maintains hierarchical relationships from the Table of
Contents (TOC) YAML. Logging is standardized and all major operations
are traced for debugging.

\begin{center}\rule{0.5\linewidth}{0.5pt}\end{center}

\subsection{Functions}\label{functions}

\subsubsection{batch\_read\_files}\label{batch_read_files}

\begin{Shaded}
\begin{Highlighting}[]
\KeywordTok{def}\NormalTok{ batch\_read\_files(file\_paths)}
\end{Highlighting}
\end{Shaded}

Read multiple files and return their contents as a dictionary
\texttt{\{path:\ content\}}. Supports \texttt{.md}, \texttt{.ipynb},
\texttt{.docx}, and other file types.

\textbf{Parameters} - \texttt{file\_paths} (list{[}str{]}): List of file
paths.

\textbf{Returns} - \texttt{dict}: Mapping of file paths to contents.

\begin{center}\rule{0.5\linewidth}{0.5pt}\end{center}

\subsubsection{read\_markdown\_file}\label{read_markdown_file}

\begin{Shaded}
\begin{Highlighting}[]
\KeywordTok{def}\NormalTok{ read\_markdown\_file(path)}
\end{Highlighting}
\end{Shaded}

Read a Markdown file and return its content as a string.

\textbf{Parameters} - \texttt{path} (str): Path to the Markdown file.

\textbf{Returns} - \texttt{str} or \texttt{None}: File content or
\texttt{None} on error.

\begin{center}\rule{0.5\linewidth}{0.5pt}\end{center}

\subsubsection{read\_notebook\_file}\label{read_notebook_file}

\begin{Shaded}
\begin{Highlighting}[]
\KeywordTok{def}\NormalTok{ read\_notebook\_file(path)}
\end{Highlighting}
\end{Shaded}

Read a Jupyter notebook file and return its content as a dictionary.

\textbf{Parameters} - \texttt{path} (str): Path to the notebook file.

\textbf{Returns} - \texttt{dict} or \texttt{None}: Notebook content or
\texttt{None} on error.

\begin{center}\rule{0.5\linewidth}{0.5pt}\end{center}

\subsubsection{read\_docx\_file}\label{read_docx_file}

\begin{Shaded}
\begin{Highlighting}[]
\KeywordTok{def}\NormalTok{ read\_docx\_file(path)}
\end{Highlighting}
\end{Shaded}

Read a DOCX file and return its text content as a string.

\textbf{Parameters} - \texttt{path} (str): Path to the DOCX file.

\textbf{Returns} - \texttt{str} or \texttt{None}: Text content or
\texttt{None} on error.

\begin{center}\rule{0.5\linewidth}{0.5pt}\end{center}

\subsubsection{batch\_extract\_assets}\label{batch_extract_assets}

\begin{Shaded}
\begin{Highlighting}[]
\KeywordTok{def}\NormalTok{ batch\_extract\_assets(contents\_dict, content\_type, }\OperatorTok{**}\NormalTok{kwargs)}
\end{Highlighting}
\end{Shaded}

Extract assets from multiple file contents in one pass. Returns a
dictionary \texttt{\{path:\ {[}asset\_records{]}\}}.

\textbf{Parameters} - \texttt{contents\_dict} (dict): Mapping of file
paths to contents. - \texttt{content\_type} (str): Type of content
(`markdown', `notebook', `docx', etc.). - \texttt{**kwargs}: Additional
arguments for DB connection/cursor.

\textbf{Returns} - \texttt{dict}: Mapping of file paths to lists of
asset records.

\begin{center}\rule{0.5\linewidth}{0.5pt}\end{center}

\subsubsection{extract\_linked\_files\_from\_markdown\_content}\label{extract_linked_files_from_markdown_content}

\begin{Shaded}
\begin{Highlighting}[]
\KeywordTok{def}\NormalTok{ extract\_linked\_files\_from\_markdown\_content(md\_text, page\_id}\OperatorTok{=}\VariableTok{None}\NormalTok{)}
\end{Highlighting}
\end{Shaded}

Extract asset links from Markdown text.

\textbf{Parameters} - \texttt{md\_text} (str): Markdown content. -
\texttt{page\_id} (optional): Page identifier for DB linking.

\textbf{Returns} - \texttt{list{[}dict{]}}: File record dicts for each
asset found.

\begin{center}\rule{0.5\linewidth}{0.5pt}\end{center}

\subsubsection{extract\_linked\_files\_from\_notebook\_cell\_content}\label{extract_linked_files_from_notebook_cell_content}

\begin{Shaded}
\begin{Highlighting}[]
\KeywordTok{def}\NormalTok{ extract\_linked\_files\_from\_notebook\_cell\_content(cell, nb\_path}\OperatorTok{=}\VariableTok{None}\NormalTok{)}
\end{Highlighting}
\end{Shaded}

Extract asset links from a notebook cell.

\textbf{Parameters} - \texttt{cell} (dict): Notebook cell. -
\texttt{nb\_path} (str, optional): Notebook file path.

\textbf{Returns} - \texttt{list{[}dict{]}}: File record dicts for each
asset found.

\begin{center}\rule{0.5\linewidth}{0.5pt}\end{center}

\subsubsection{extract\_linked\_files\_from\_docx\_content}\label{extract_linked_files_from_docx_content}

\begin{Shaded}
\begin{Highlighting}[]
\KeywordTok{def}\NormalTok{ extract\_linked\_files\_from\_docx\_content(docx\_path, page\_id}\OperatorTok{=}\VariableTok{None}\NormalTok{)}
\end{Highlighting}
\end{Shaded}

Extract asset links from a DOCX file.

\textbf{Parameters} - \texttt{docx\_path} (str): Path to DOCX file. -
\texttt{page\_id} (optional): Page identifier for DB linking.

\textbf{Returns} - \texttt{list{[}dict{]}}: File record dicts for each
asset found.

\begin{center}\rule{0.5\linewidth}{0.5pt}\end{center}

\subsubsection{populate\_site\_info\_from\_config}\label{populate_site_info_from_config}

\begin{Shaded}
\begin{Highlighting}[]
\KeywordTok{def}\NormalTok{ populate\_site\_info\_from\_config(config\_filename}\OperatorTok{=}\StringTok{\textquotesingle{}\_config.yml\textquotesingle{}}\NormalTok{)}
\end{Highlighting}
\end{Shaded}

Populate the \texttt{site\_info} table from the given config file.

\textbf{Parameters} - \texttt{config\_filename} (str): Name of the
config file (default: '\_config.yml').

\begin{center}\rule{0.5\linewidth}{0.5pt}\end{center}

\subsubsection{get\_conversion\_flags}\label{get_conversion_flags}

\begin{Shaded}
\begin{Highlighting}[]
\KeywordTok{def}\NormalTok{ get\_conversion\_flags(extension)}
\end{Highlighting}
\end{Shaded}

Get conversion capability flags for a given file extension using the
database.

\textbf{Parameters} - \texttt{extension} (str): File extension (e.g.,
`.md', `.ipynb').

\textbf{Returns} - \texttt{dict}: Conversion capability flags.

\begin{center}\rule{0.5\linewidth}{0.5pt}\end{center}

\subsubsection{scan\_toc\_and\_populate\_db}\label{scan_toc_and_populate_db}

\begin{Shaded}
\begin{Highlighting}[]
\KeywordTok{def}\NormalTok{ scan\_toc\_and\_populate\_db(config\_path)}
\end{Highlighting}
\end{Shaded}

Walk the TOC from the config YAML, read each file, extract
assets/images, and populate the DB with content and asset records.
Maintains TOC hierarchy and section relationships.

\textbf{Parameters} - \texttt{config\_path} (str): Path to the config
YAML file.

\begin{center}\rule{0.5\linewidth}{0.5pt}\end{center}

\subsubsection{get\_descendants\_for\_parent}\label{get_descendants_for_parent}

\begin{Shaded}
\begin{Highlighting}[]
\KeywordTok{def}\NormalTok{ get\_descendants\_for\_parent(parent\_output\_path, db\_path)}
\end{Highlighting}
\end{Shaded}

Query all children, grandchildren, and deeper descendants for a given
parent section using a recursive CTE.

\textbf{Parameters} - \texttt{parent\_output\_path} (str): Output path
of the parent section. - \texttt{db\_path} (str): Path to the SQLite
database.

\textbf{Returns} - \texttt{list{[}dict{]}}: Dicts for each descendant
(id, title, output\_path, parent\_output\_path, slug, level).

\begin{center}\rule{0.5\linewidth}{0.5pt}\end{center}

\subsection{Usage Example}\label{usage-example}

\begin{Shaded}
\begin{Highlighting}[]
\ImportTok{from}\NormalTok{ oerforge }\ImportTok{import}\NormalTok{ scan}
\NormalTok{scan.scan\_toc\_and\_populate\_db(}\StringTok{\textquotesingle{}\_content.yml\textquotesingle{}}\NormalTok{)}
\end{Highlighting}
\end{Shaded}

\begin{center}\rule{0.5\linewidth}{0.5pt}\end{center}

\subsection{Requirements}\label{requirements}

\begin{itemize}
\tightlist
\item
  Python 3.7+
\item
  SQLite3
\item
  PyYAML
\item
  python-docx
\end{itemize}

\begin{center}\rule{0.5\linewidth}{0.5pt}\end{center}

\subsection{See Also}\label{see-also}

\begin{itemize}
\tightlist
\item
  \href{https://python-markdown.github.io/}{Python-Markdown}
\item
  \href{https://nbformat.readthedocs.io/en/latest/}{Jupyter Notebook
  Format}
\item
  \href{https://python-docx.readthedocs.io/en/latest/}{python-docx
  Documentation}
\end{itemize}

\begin{center}\rule{0.5\linewidth}{0.5pt}\end{center}

\subsection{License}\label{license}

See \texttt{LICENSE} in the project root.
