\section{build.py --- OERForge Build
Orchestrator}\label{build.py-oerforge-build-orchestrator}

\subsection{Overview}\label{overview}

\texttt{build.py} is the main workflow orchestrator for the OERForge
static site generator. It coordinates the entire build process,
including database initialization, content scanning, conversion, HTML
generation, exporting, and deployment preparation. This script is
designed for new users and programmers to provide a clear, step-by-step
build pipeline for open educational resources.

\begin{itemize}
\tightlist
\item
  \textbf{Initializes the content database}
\item
  \textbf{Scans the table of contents and populates the database}
\item
  \textbf{Converts all source content to output formats}
\item
  \textbf{Exports all content and assets to the build directory}
\item
  \textbf{Builds HTML pages and section indexes}
\item
  \textbf{Copies build outputs to docs/ for publishing}
\item
  \textbf{Logs all major steps and errors}
\end{itemize}

\begin{center}\rule{0.5\linewidth}{0.5pt}\end{center}

\subsection{Functions}\label{functions}

\subsubsection{run()}\label{run}

Runs the complete OERForge build workflow.

\textbf{Parameters} - None

\textbf{Returns} - None

\textbf{Workflow Steps} 1. Initializes the SQLite database for content
and assets. 2. Scans the table of contents (\texttt{\_content.yml}) and
populates the database. 3. Converts all source content (Markdown,
Jupyter, DOCX) to output formats. 4. Exports all content and assets to
the build directory. 5. Builds HTML pages and section indexes using
templates. 6. Copies the build output to \texttt{docs/} for publishing
(e.g., GitHub Pages). 7. Logs all steps and errors to the build log.

\begin{center}\rule{0.5\linewidth}{0.5pt}\end{center}

\subsection{Constants}\label{constants}

\begin{itemize}
\tightlist
\item
  \texttt{BUILD\_DIR}: Path to the build output directory.
\item
  \texttt{FILES\_DIR}: Subdirectory for files in build.
\item
  \texttt{PROJECT\_ROOT}: Absolute path to the project root.
\item
  \texttt{BUILD\_FILES\_DIR}: Path to files in build.
\item
  \texttt{BUILD\_HTML\_DIR}: Path to HTML output in build.
\end{itemize}

\begin{center}\rule{0.5\linewidth}{0.5pt}\end{center}

\subsection{Logging}\label{logging}

All major operations and errors are logged for debugging and auditing.
The build log can be found in the \texttt{log/} directory.

\subsection{Error Handling}\label{error-handling}

Robust error handling is implemented in each step of the workflow
modules. All failures are logged with context.

\subsection{Example Usage}\label{example-usage}

\begin{Shaded}
\begin{Highlighting}[]
\CommentTok{\# Run the build workflow from the command line}
\NormalTok{python build.py}

\CommentTok{\# Or import and run from another script}
\ImportTok{from}\NormalTok{ build }\ImportTok{import}\NormalTok{ run}
\NormalTok{run()}
\end{Highlighting}
\end{Shaded}

\begin{center}\rule{0.5\linewidth}{0.5pt}\end{center}

\subsection{License}\label{license}

See LICENSE in the project root.
