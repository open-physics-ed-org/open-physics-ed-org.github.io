\section{oerforge.make}\label{oerforge.make}

Hugo-style Markdown to HTML Static Site Generator (Python)

\begin{center}\rule{0.5\linewidth}{0.5pt}\end{center}

\subsection{Overview}\label{overview}

\texttt{oerforge.make} provides functions for building static HTML sites
from Markdown using Jinja2 templates, asset management, navigation, and
download button generation. It is inspired by Hugo and designed for
clarity, maintainability, and extensibility.

\begin{center}\rule{0.5\linewidth}{0.5pt}\end{center}

\subsection{Functions}\label{functions}

\subsubsection{copy\_static\_assets\_to\_build}\label{copy_static_assets_to_build}

\begin{Shaded}
\begin{Highlighting}[]
\KeywordTok{def}\NormalTok{ copy\_static\_assets\_to\_build()}
\end{Highlighting}
\end{Shaded}

Copy static assets (CSS, JS, images) from \texttt{static/} to
\texttt{build/}. Overwrites files each time it is called.

\begin{center}\rule{0.5\linewidth}{0.5pt}\end{center}

\subsubsection{get\_available\_downloads\_for\_page}\label{get_available_downloads_for_page}

\begin{Shaded}
\begin{Highlighting}[]
\KeywordTok{def}\NormalTok{ get\_available\_downloads\_for\_page(rel\_path, page\_dir}\OperatorTok{=}\VariableTok{None}\NormalTok{)}
\end{Highlighting}
\end{Shaded}

Scan the published output directory for a page and return a list of
available download formats.

\textbf{Parameters} - \texttt{rel\_path} (str): Relative path to the
HTML file (e.g., `about/index.html'). - \texttt{page\_dir} (str,
optional): Directory containing downloadable files.

\textbf{Returns} - \texttt{list{[}dict{]}}: List of available downloads
with label, filename, href, theme, and aria\_label.

\begin{center}\rule{0.5\linewidth}{0.5pt}\end{center}

\subsubsection{build\_download\_buttons\_context}\label{build_download_buttons_context}

\begin{Shaded}
\begin{Highlighting}[]
\KeywordTok{def}\NormalTok{ build\_download\_buttons\_context(rel\_path, page\_dir}\OperatorTok{=}\VariableTok{None}\NormalTok{)}
\end{Highlighting}
\end{Shaded}

Build the download buttons context for a page.

\textbf{Parameters} - \texttt{rel\_path} (str): Relative path to the
HTML file. - \texttt{page\_dir} (str, optional): Directory containing
downloadable files.

\textbf{Returns} - \texttt{list{[}dict{]}}: List of button dictionaries
for the template.

\begin{center}\rule{0.5\linewidth}{0.5pt}\end{center}

\subsubsection{slugify}\label{slugify}

\begin{Shaded}
\begin{Highlighting}[]
\KeywordTok{def}\NormalTok{ slugify(title: }\BuiltInTok{str}\NormalTok{) }\OperatorTok{{-}\textgreater{}} \BuiltInTok{str}
\end{Highlighting}
\end{Shaded}

Convert a title to a slug suitable for folder names.

\textbf{Parameters} - \texttt{title} (str): Page or section title.

\textbf{Returns} - \texttt{str}: Slugified string.

\begin{center}\rule{0.5\linewidth}{0.5pt}\end{center}

\subsubsection{load\_yaml\_config}\label{load_yaml_config}

\begin{Shaded}
\begin{Highlighting}[]
\KeywordTok{def}\NormalTok{ load\_yaml\_config(config\_path: }\BuiltInTok{str}\NormalTok{) }\OperatorTok{{-}\textgreater{}} \BuiltInTok{dict}
\end{Highlighting}
\end{Shaded}

Load and parse the YAML config file.

\textbf{Parameters} - \texttt{config\_path} (str): Path to the YAML
config file.

\textbf{Returns} - \texttt{dict}: Parsed configuration data.

\begin{center}\rule{0.5\linewidth}{0.5pt}\end{center}

\subsubsection{ensure\_output\_dir}\label{ensure_output_dir}

\begin{Shaded}
\begin{Highlighting}[]
\KeywordTok{def}\NormalTok{ ensure\_output\_dir(md\_path)}
\end{Highlighting}
\end{Shaded}

Ensure the output directory for the HTML file exists, mirroring
\texttt{build/files} structure.

\textbf{Parameters} - \texttt{md\_path} (str): Path to the Markdown
file.

\begin{center}\rule{0.5\linewidth}{0.5pt}\end{center}

\subsubsection{setup\_template\_env}\label{setup_template_env}

\begin{Shaded}
\begin{Highlighting}[]
\KeywordTok{def}\NormalTok{ setup\_template\_env()}
\end{Highlighting}
\end{Shaded}

Set up the Jinja2 template environment for rendering pages.

\textbf{Returns} - \texttt{jinja2.Environment}: Configured Jinja2
environment.

\begin{center}\rule{0.5\linewidth}{0.5pt}\end{center}

\subsubsection{render\_page}\label{render_page}

\begin{Shaded}
\begin{Highlighting}[]
\KeywordTok{def}\NormalTok{ render\_page(context: }\BuiltInTok{dict}\NormalTok{, template\_name: }\BuiltInTok{str}\NormalTok{) }\OperatorTok{{-}\textgreater{}} \BuiltInTok{str}
\end{Highlighting}
\end{Shaded}

Render a page using Hugo-style templates.

\textbf{Parameters} - \texttt{context} (dict): Context dictionary for
the template. - \texttt{template\_name} (str): Name of the template
file.

\textbf{Returns} - \texttt{str}: Rendered HTML string.

\begin{center}\rule{0.5\linewidth}{0.5pt}\end{center}

\subsubsection{generate\_nav\_menu}\label{generate_nav_menu}

\begin{Shaded}
\begin{Highlighting}[]
\KeywordTok{def}\NormalTok{ generate\_nav\_menu(context: }\BuiltInTok{dict}\NormalTok{) }\OperatorTok{{-}\textgreater{}} \BuiltInTok{list}
\end{Highlighting}
\end{Shaded}

Generate top-level navigation menu items from the content table.

\textbf{Parameters} - \texttt{context} (dict): Context dictionary,
typically with \texttt{rel\_path}.

\textbf{Returns} - \texttt{list{[}dict{]}}: List of menu item
dictionaries.

\begin{center}\rule{0.5\linewidth}{0.5pt}\end{center}

\subsubsection{get\_header\_partial}\label{get_header_partial}

\begin{Shaded}
\begin{Highlighting}[]
\KeywordTok{def}\NormalTok{ get\_header\_partial(context: }\BuiltInTok{dict}\NormalTok{) }\OperatorTok{{-}\textgreater{}} \BuiltInTok{str}
\end{Highlighting}
\end{Shaded}

Render the header partial using Jinja2.

\textbf{Parameters} - \texttt{context} (dict): Context dictionary for
the template.

\textbf{Returns} - \texttt{str}: Rendered header HTML.

\begin{center}\rule{0.5\linewidth}{0.5pt}\end{center}

\subsubsection{get\_footer\_partial}\label{get_footer_partial}

\begin{Shaded}
\begin{Highlighting}[]
\KeywordTok{def}\NormalTok{ get\_footer\_partial(context: }\BuiltInTok{dict}\NormalTok{) }\OperatorTok{{-}\textgreater{}} \BuiltInTok{str}
\end{Highlighting}
\end{Shaded}

Render the footer partial using Jinja2.

\textbf{Parameters} - \texttt{context} (dict): Context dictionary for
the template.

\textbf{Returns} - \texttt{str}: Rendered footer HTML.

\begin{center}\rule{0.5\linewidth}{0.5pt}\end{center}

\subsubsection{convert\_markdown\_to\_html}\label{convert_markdown_to_html}

\begin{Shaded}
\begin{Highlighting}[]
\KeywordTok{def}\NormalTok{ convert\_markdown\_to\_html(md\_path: }\BuiltInTok{str}\NormalTok{) }\OperatorTok{{-}\textgreater{}} \BuiltInTok{str}
\end{Highlighting}
\end{Shaded}

Convert Markdown to HTML using markdown-it-py, rewriting local image
paths and adding accessibility roles.

\textbf{Parameters} - \texttt{md\_path} (str): Path to the Markdown
file.

\textbf{Returns} - \texttt{str}: Rendered HTML string.

\begin{center}\rule{0.5\linewidth}{0.5pt}\end{center}

\subsubsection{convert\_markdown\_to\_html\_text}\label{convert_markdown_to_html_text}

\begin{Shaded}
\begin{Highlighting}[]
\KeywordTok{def}\NormalTok{ convert\_markdown\_to\_html\_text(md\_text: }\BuiltInTok{str}\NormalTok{) }\OperatorTok{{-}\textgreater{}} \BuiltInTok{str}
\end{Highlighting}
\end{Shaded}

Convert Markdown text to HTML using markdown-it-py, rewriting local
image paths and adding accessibility roles.

\textbf{Parameters} - \texttt{md\_text} (str): Markdown text.

\textbf{Returns} - \texttt{str}: Rendered HTML string.

\begin{center}\rule{0.5\linewidth}{0.5pt}\end{center}

\subsubsection{get\_asset\_path}\label{get_asset_path}

\begin{Shaded}
\begin{Highlighting}[]
\KeywordTok{def}\NormalTok{ get\_asset\_path(asset\_type, asset\_name, rel\_path)}
\end{Highlighting}
\end{Shaded}

Compute the relative asset path for CSS, JS, or images based on page
depth.

\textbf{Parameters} - \texttt{asset\_type} (str): Asset type (`css',
`js', `images'). - \texttt{asset\_name} (str): Asset filename. -
\texttt{rel\_path} (str): Relative path to the page.

\textbf{Returns} - \texttt{str}: Relative asset path.

\begin{center}\rule{0.5\linewidth}{0.5pt}\end{center}

\subsubsection{add\_asset\_paths}\label{add_asset_paths}

\begin{Shaded}
\begin{Highlighting}[]
\KeywordTok{def}\NormalTok{ add\_asset\_paths(context, rel\_path)}
\end{Highlighting}
\end{Shaded}

Add asset paths (CSS, JS, logo) to the context for template rendering.

\textbf{Parameters} - \texttt{context} (dict): Context dictionary. -
\texttt{rel\_path} (str): Relative path to the page.

\textbf{Returns} - \texttt{dict}: Updated context dictionary.

\begin{center}\rule{0.5\linewidth}{0.5pt}\end{center}

\subsubsection{get\_top\_level\_sections}\label{get_top_level_sections}

\begin{Shaded}
\begin{Highlighting}[]
\KeywordTok{def}\NormalTok{ get\_top\_level\_sections(db\_path}\OperatorTok{=}\VariableTok{None}\NormalTok{)}
\end{Highlighting}
\end{Shaded}

Get all top-level sections from the database for section index
generation.

\textbf{Parameters} - \texttt{db\_path} (str, optional): Path to the
SQLite database.

\textbf{Returns} - \texttt{list{[}tuple{]}}: List of (section\_title,
output\_dir) tuples.

\begin{center}\rule{0.5\linewidth}{0.5pt}\end{center}

\subsubsection{build\_section\_indexes}\label{build_section_indexes}

\begin{Shaded}
\begin{Highlighting}[]
\KeywordTok{def}\NormalTok{ build\_section\_indexes()}
\end{Highlighting}
\end{Shaded}

Generate index.html files for all top-level sections using the section
template.

\begin{center}\rule{0.5\linewidth}{0.5pt}\end{center}

\subsubsection{build\_all\_markdown\_files}\label{build_all_markdown_files}

\begin{Shaded}
\begin{Highlighting}[]
\KeywordTok{def}\NormalTok{ build\_all\_markdown\_files()}
\end{Highlighting}
\end{Shaded}

Build all Markdown files using Hugo-style rendering, using the first \#
header as the title.

\begin{center}\rule{0.5\linewidth}{0.5pt}\end{center}

\subsubsection{create\_section\_index\_html}\label{create_section_index_html}

\begin{Shaded}
\begin{Highlighting}[]
\KeywordTok{def}\NormalTok{ create\_section\_index\_html(section\_title: }\BuiltInTok{str}\NormalTok{, output\_dir: }\BuiltInTok{str}\NormalTok{, context: }\BuiltInTok{dict}\NormalTok{)}
\end{Highlighting}
\end{Shaded}

Generate section \texttt{index.html} using the \texttt{section.html}
template.

\textbf{Parameters} - \texttt{section\_title} (str): Title of the
section. - \texttt{output\_dir} (str): Output directory for the section
index. - \texttt{context} (dict): Context dictionary for the template.

\begin{center}\rule{0.5\linewidth}{0.5pt}\end{center}

\subsection{Usage Example}\label{usage-example}

\begin{Shaded}
\begin{Highlighting}[]
\ImportTok{from}\NormalTok{ oerforge }\ImportTok{import}\NormalTok{ make}
\NormalTok{make.build\_section\_indexes()}
\NormalTok{make.build\_all\_markdown\_files()}
\end{Highlighting}
\end{Shaded}

\begin{center}\rule{0.5\linewidth}{0.5pt}\end{center}

\subsection{Requirements}\label{requirements}

\begin{itemize}
\tightlist
\item
  Python 3.7+
\item
  Jinja2
\item
  markdown-it-py
\item
  mdit-py-plugins
\item
  PyYAML
\item
  SQLite3
\end{itemize}

\begin{center}\rule{0.5\linewidth}{0.5pt}\end{center}

\subsection{See Also}\label{see-also}

\begin{itemize}
\tightlist
\item
  \href{https://jinja.palletsprojects.com/}{Jinja2 Documentation}
\item
  \href{https://markdown-it-py.readthedocs.io/}{markdown-it-py
  Documentation}
\end{itemize}

\begin{center}\rule{0.5\linewidth}{0.5pt}\end{center}

\subsection{License}\label{license}

See \texttt{LICENSE} in the project root.
